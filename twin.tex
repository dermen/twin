\documentclass [11pt,fleqn]{article}

\usepackage{amssymb}
%\usepackage{epsf,psfig,graphicx}
%\usepackage{epstopdf} % TJL ADDED
\usepackage{bm}
\usepackage{epsf,graphicx}
\usepackage{psfrag}
\usepackage{amsmath}
\usepackage[mathscr]{euscript}
\usepackage{color}
\topmargin     -0.60in  % (adjusted for printer bias) 
\headheight      .00in  % (no headers) 
\headsep         .50in  % (top margin + headers + skip) 
\textheight     9.50in  % (instructions: 9 1/8'' min, 9 7/16'' max) 
\textwidth      6.00in  % 2*3.33 + .33 = 6.99 
\oddsidemargin  0.3125in  % (subtracted 1inch bias) 
\evensidemargin 0.3125in 
%\renewcommand{\baselinestretch}{1.5} 
\parindent .0in 
\parskip 10pt 


\font \bigtenrm=cmmi10 scaled\magstep2 
\def \dt {\delta\tau} 
\def \ve {\varepsilon} 
\def \ch {{\cal H}} 
\def \del {\partial} 
\def \be {\begin{equation}} 
\def \ee {\end{equation}} 
\def \beq {\begin{eqnarray}} 
\def \eeq {\end{eqnarray}} 
\def \tv {\tilde v} 
\def \veren {\varepsilon^{\rm ren}_f} 
\def \vef {\varepsilon^0_f} 
\def \su {\uparrow} 
\def \sd {\downarrow} 
\def \CR {\nonumber\\} 
\def \hfb {\hfill\break} 
\def \tb {\bar{t} } 
\def \kb {\bar{k} } 
\def \tbB {\bar{t}_B } 
\renewcommand*{\thefootnote}{\fnsymbol{footnote}}
%\def \ul{#1} {$\underline{#1 }$}

\usepackage{authblk}
\usepackage{lineno}
\title{Nanoparticle twinning observed using correlated x-ray scattering with solution samples}
%\author[1]{ Derek Mendez }
%\author[3]{ Thomas J. Lane}
%\author[1,2]{Jongmin Sung}
%\author[7]{Jonas Sellberg}
%\author[4,6]{Cl\'ement Levard}
%\author[1]{Herschel Watkins}
%\author[7]{Aina E. Cohen}
%\author[7]{Michael Soltis}
%\author[2,5]{Shirley Sutton}
%\author[2]{James Spudich}
%\author[3]{Vijay Pande}
%\author[7]{Daniel Ratner}
%\author[1,7]{Sebastian Doniach} %\thanks{corresponding author: sxdwc@slac.stanford.edu}}

%\affil[1]{Stanford University Department of Applied Physics, Stanford, CA 94305}
%\affil[2]{Stanford University Department of Biochemistry, Stanford, CA 94305}
%\affil[3]{Stanford University Department of Chemistry, Stanford, CA 94305}
%\affil[4]{Stanford University Department of Geological and Environmental Sciences, Stanford, CA 94305}
%\affil[5]{Stanford University School of Medicine, Stanford, CA 94305}
%\affil[6]{Aix-Marseille Universit\'e, CNRS, IRD, CEREGE UM34, 13545 Aix en Provence, France}
%\affil[7]{SLAC National Accelerator Laboratory, Menlo Park, CA 94025}

%\renewcommand\Authands{ and }
\author{Doniach group}
\date{}
\begin{document} 
%\setpagewiselinenumbers
%\linenumbers
\maketitle

\delimitershortfall=-1pt

%{\bf Abstract}


%{\bf Keywords/phrases} 

\section{Introduction}


%x-ray angular correlations, Bragg peak correlations, silver nanoparticles, atomic resolution x-ray scattering, synchrotron radiation, solution ensemble
\section{CXS overview}

Correlated x-ray scattering (CXS) is an emerging technique that aims to extract structural information from solution scattering measurements. We define a solution as a set of identical, non-interacting objects $m$, each with an independent orientation $\omega$ relative to the x-ray beam axis, governed by the objects diffusion constant. We consider $m$ as a collection of $N_a$ atoms each with position vector 
\be
\bm r^m_j (t) = \bm R^m_\omega (t)\cdot \bm r_j + \bm T^m(t) \qquad 1 \le j \le N_a
\ee 
where $\bm R^m_\omega(t)$ is a rotation operator, $\bm T^m (t)$ is a translation operator representing the center of mass position of $m$ at time $t$, and $\bm r_j$ is the position of the $j^{th}$ atom at an arbitrarily defined initial orientation. If we were to freeze the solution at an instant in time and expose it to x-ray photons of wavelength $\lambda$, then we could measure the scattering function

\be
S(\bm q, t) = \left | \sum_m \sum_{j}^{N_a} f_j(q) \,e ^ { -i \,\bm q \cdot \bm  r^m _j (t)  } \right |^2
\ee

Here $f_j( q  )$ is the atomic form factor of the $j^{th}$ atom$, \bm q$ represents a position in reciprocal space (e.g. of a pixel) at scattering angle  

\be \label{angle}
\sin(\theta) = \frac{ \lambda }{ 4 \pi}\,q
\ee

and the outer sum is over all exposed $m$. In general, $S(\bm q, t)$ may be written as

\be \label{ sums }
S( \bm q, t) = \sum_m \left | A^m_\omega (\bm q ) \right|^2 + \sum _{ m \neq m' } A^m_\omega (\bm q) \left (A^{m'}_\omega (\bm q) \right )^*
\ee

where 

\be
A^m_\omega (\bm q) = \sum_{j}^{N_a} f_j(q) e^{ -i \,\bm q \cdot \bm  r^m_j (t)  }
\ee

The strength of the interference term on the RHS of (\ref{ sums }) depends on both the concentration of the sample, and the magnitude of the scattering angle (\ref{angle}). In what follows we will work at relatively high scattering angles, hence we can safely neglect this limit such that we have

\beq \label{ scatter}
S( \bm q, t) &=& \sum_m \left | A^m_\omega (\bm q) \right|^2 \\
&=& \sum_m \left | \sum_{j}^{N_a} f_j(q)\, e^ { -i \,\bm q \cdot \bm  r^m _j (t)} \right | ^2   \\
&=& \sum_m \left | \sum_{j}^{N_a} f_j(q)\, e^ { -i \,\bm q \cdot \left (\bm {R}^m_\omega \left(t\right)\cdot \bm r_j \,+\, \bm {T}^m\left(t\right) \right)} \right | ^2 \\
&=& \sum_m \left | \sum_{j}^{N_a} f_j(q)\, e^ { -i \,\bm q \cdot \bm {R}^m_\omega (t)\cdot \bm r_j }  \right | ^2 
\eeq

Upon exposing the solution for a finite time, we can obtain total scattered photons

\be
n_s( \bm q ) \propto \int_{\tau} dt \,S( \bm q, t )
\ee

where $s$ is used to denote a unique exposure and $\tau$ is the exposure duration. For sufficiently long $\tau$, the number of photons scattered from particles in each orientation $\omega$ becomes a constant and we have a well defined average value 
%\be
%\int_{\tau} dt \sum_m  \xrightarrow{\tau \rightarrow \infty} \frac{N_m}{8\pi^2}\int d\omega
%\ee
%and we have
\be 
\bar n(q) \propto \int d\omega \left | A_\omega (\bm q) \right | ^2
\ee

However, provided $\tau$ is much shorter than the characteristic diffusion time, one can measure a finite fluctuation of the scattering

\be
\delta n_s(\bm q) = n_s (\bm q) - \bar n (q )
\ee

Each additional exposure will contain 

During solution exposures, correlations arise whenever a given particle scatters photons into two pixels on a detection apparatus. These pixels are defined by unique scattering vectors $\bm q_1$ and $\bm q_2$. and the angle

\be
\cos (\psi) = \frac{\bm q_1 \cdot \bm q_2}{|\bm q_1| |\bm q_2| } 
\ee

is constrained by that particle's internal structure. Successful extraction of these structural constraints can enhance the problem of model refinement and prediction using solution measurements [refs]. From many snapshot exposures of solutions, it was theorized [refs] that one could extract a correlation function

\be \label{corr}
C(\bm q_1, \bm q_2) \equiv C(|\bm q_1|,| \bm q_2|,\cos (\psi) ) 
\ee

dependent only on the average internal structure of the particles in solution [refs]. The overall shape of (\ref{corr}) depends on the angular constraints aforementioned: the magnitude of (\ref{corr}) will rise and fall depending on the local electronic density of the scatterers. The establishment of protocols for successfully measuring and extracting CXS signal from solution measurements has potential to revolutionize atomic imaging disciplines. It has been shown that super-resolution can be achieved using iterative algorithms which refine low-resolution models using detailed x-ray data [refs] and CXS data are extremely detailed owing to the three-dimensional parameter space. 

The signal-to-noise ratio for (\ref{corr}) increases with each additional recorded exposure, and was proven to be independent of the number of particles in the exposed sample volume [refs]. If one can accurately separate CXS from artificial correlations in the experiment, then in principle measuring CXS of weak scatterers such as proteins in solution should be possible; one need only record enough exposures. With state-of-the-art x-ray free electron laser (xFEL) facilities one can scatter enough photons per protein in order to make such an experiment feasible [refs]. \emph{(include numbers here on protein scattering cross sections and xFEL focus/flux, showing that one can achieve, say 2+ photons per protein per exposure on average. Perhaps we can use our LCLS data to show this?)}

Currently, the main inhibitions to accurate CXS measurement are fluctuations in the measured intensity not due to the particles under investigation. These arise as artifacts associated with the experiment, including non-uniform pixel response, non-uniform particle distribution in the solution, and background (solvent) scattering [refs]. All of these can lead to artificial intensity correlations. In order to build a foundation for doing CXS on solutions, and to establish methods for overcoming the associated challenges, we have been conducting CXS experiments involving nanoparticles (NPs) in solution. NPs generally have high scattering cross section, making them an good test candidate.

Previously, we reported on extracting CXS from silver NP solution at the Stanford Synchrotron Radiation Lightsource (SSRL). There we made use of a non-linear binary filter and were able to accurately extract CXS from the data which was plagued with artificial correlations. Here we report on substantial advances in our analysis framework, which have allowed us to circumvent the binary filtering step and further resolve additional correlations than those expected from elementary bulk face-centered-cubic (FCC) models of silver NPs. A second experiment involving gold NPs at the xFEL in Japan  confirmed these results, and we have determined that the additional correlations arise due to a very simple NP twinning model [refs]. As CXS from solutions reveals information on the average particle in solution, this implies that the twinning model is common to most FCC NPs in our experiments. By analyzing the magnitude of (\ref{corr}) we estimate the percentage of each individual in the twins.

With these results we demonstrate how CXS can extract additional structural motifs not readily apparent with conventional solution scattering methods.
 
\section{CXS of nanoparticles in solution}

An obvious precursor to the study of soft matter  CXS is that of nanoparticles (NPs). Indeed our first successful measurement of CXS was done on colloidal suspensions of silver NPs at SSRL [refs]. A crystalline NP scatters photons into discrete vectors $\bm G_{hkl}$ defined by a set of crystallographic planes, where $hkl$ are the miller indices [refs]. Such an NP will scatter photons into the detector provided it is oriented accordingly. If we define a rotation operator $\hat{\bm R}_\omega$ with $\omega$ a triple of Euler angles specifying the NP orientation, and we define the detector as the set of pixels $\{\bm q\}$, then an NP at orientation $\omega$ can scatter into the detector provided 

\be \label{condition}
\hat{\bm R}_\omega \cdot \bm G_{hkl} \in \{\bm q\}
\ee

In order to measure (\ref{corr}), a fraction of NPs in solution must be oriented such that condition (\ref{condition}) is met for two scattering vectors, $\bm G_{hkl}$ and $\bm G_{h'k'l'}$. If enough photons scatter into both vectors, then one can detect angular correlations at $\psi$ corresponding to the interplanar angles between sets of crystallographic planes $hkl$ and $h'k'l'$. As an example, consider the $\{111\}$ family of planes for FCC metals like silver: angular correlations will arise at

\be
\cos (\psi) = \frac{\bm G_{111} \cdot \bm G_{11\bar{1}}}{|\bm G_{111}| |\bm G_{11\bar{1}}| } = \frac{1}{3}
\ee

and

\be
\cos (\psi) = \frac{\bm G_{111} \cdot \bm G_{\bar{1}\bar{1}1}}{|\bm G_{111}| |\bm G_{\bar{1}\bar{1}1}| } = -\frac{1}{3}
\ee

Note that these are supplementary angles, which is a direct consequence of Friedel symmetry [refs] between $G_{11\bar{1}}$ and $G_{\bar{1}\bar{1}1}$. Therefore, correlations of should be observed in supplementary angle pairs.  Table 1 lists the expected CXS correlations from homogenous FCC NPs.

\section{Results}



%Similarly, correlations will arise from scattering between $\{111\}$ and $\{200\}$ planes at angles given by

%\be
%cos(\psi) = \frac{\bm G_{111} \cdot \bm G_{200}} {|\bm G_{111}| |\bm G_{200| }} = \frac{1}{\sqrt{3}}
%\ee

%and

%\be
%cos(\psi) = \frac{\bm G_{\bar{1}11} \cdot \bm G_{200}}{|\bm G_{\bar{1}11}| |\bm G_{200| }} = -\frac{1}{\sqrt{3}} 
%\ee

%Even more trivially, for the $\{200\}$ set of planes, correlations will arise at $cos(\psi) = 0$, as the interplanar angles between $\{200\}$.

%\be
%cos(\psi) = \frac {\bm G_{200} \cdot \bm G_{020}} {|\bm G_{200}| |\bm G_{020}|} = 0
%\ee




%\begin{figure}
%\begin{center}
%\includegraphics[ ]{./setup.eps}
%\end{center}
%\caption{Experimental setup: a kapton capillary filled with a solution of silver NPs (face-centered-cubic). Bragg rings $q_{111}$ and $q_{200}$ are illustrated by circles on the detector plane. At least one of the exposed NPs happens to be oriented such that two reciprocal lattice (body-centered-cubic) peaks are intersecting the detector at $q_{111}$. Dashed lines represent the scattering vectors (separated by the angle $\psi$), and solid lines represent the projection of those vectors onto the detector plane (separated by the angle $\Delta$). Artwork courtesy of Gregory M.~Stewart (SLAC).}
%\label{fig:setup}
%\end{figure}

\bibliographystyle{prsb}
%\bibliography{papers2.bib}
%
%
%
\begin{thebibliography}{10}
\expandafter\ifx\csname urlstyle\endcsname\relax
  \providecommand{\doi}[1]{doi:\discretionary{}{}{}#1}\else
  \providecommand{\doi}{doi:\discretionary{}{}{}\begingroup
  \urlstyle{rm}\Url}\fi

\bibitem{Kam:1977wc}
Kam, Z., 1977 {Determination of macromolecular structure in solution by spatial
  correlation of scattering fluctuations}.
\newblock \emph{Macromolecules} \textbf{10}, 927--934.

\bibitem{Saldin:2011ch}
Saldin, D., Poon, H., Bogan, M., Marchesini, S., Shapiro, D., Kirian, R.,
  Weierstall, U. \& Spence, J., 2011 {New Light on Disordered Ensembles: Ab
  Initio Structure Determination of One Particle from Scattering Fluctuations
  of Many Copies}.
\newblock \emph{Phys. Rev. Lett.} \textbf{106}, 115501.

\bibitem{Kam:1981ua}
Kam, Z., Koch, M.~H. \& Bordas, J., 1981 {Fluctuation x-ray scattering from
  biological particles in frozen solution by using synchrotron radiation.}
\newblock \emph{Proceedings of the National Academy of Sciences} \textbf{78},
  3559--3562.

\bibitem{Wochner:2009ia}
Wochner, P., Gutt, C., Autenrieth, T., Demmer, T., Bugaev, V., Ortiz, A.~D.,
  Duri, A., Zontone, F., Gr{\"u}bel, G. \& Dosch, H., 2009 {X-ray cross
  correlation analysis uncovers hidden local symmetries in disordered matter.}
\newblock \emph{P Natl Acad Sci Usa} \textbf{106}, 11511--11514.

\bibitem{Kam:1985tz}
Kam, Z. \& Gafni, I., 1985 {Three-dimensional reconstruction of the shape of
  human wart virus using spatial correlations.}
\newblock \emph{Ultramicroscopy} \textbf{17}, 251--262.

\bibitem{Starodub:1fy}
Starodub, D., Aquila, A., Bajt, S., Barthelmess, M., Barty, A., Bostedt, C.,
  Bozek, J.~D., Coppola, N., Doak, R.~B., Epp, S.~W. \emph{et~al.}, 1
  {Single-particle structure determination by correlations of snapshot X-ray
  diffraction patterns}.
\newblock \emph{Nature Communications} \textbf{3}, 1276--7.

\bibitem{Elser:2011ez}
Elser, V., 2011 {Strategies for processing diffraction data from randomly
  oriented particles}.
\newblock \emph{Ultramicroscopy} \textbf{111}, 788--792.

\bibitem{Liu:2013dv}
Liu, H., Poon, B.~K., Saldin, D.~K., Spence, J. C.~H. \& Zwart, P.~H., 2013
  {Three-dimensional single-particle imaging using angular correlations from
  X-ray laser data}.
\newblock \emph{Acta Crystallogr A Found Crystallogr} \textbf{69}, 365--373.

\bibitem{Chen:2013io}
Chen, G., Zwart, P.~H. \& Li, D., 2013 {Component Particle Structure in
  Heterogeneous Disordered Ensembles Extracted from High-Throughput Fluctuation
  X-Ray Scattering}.
\newblock \emph{Phys. Rev. Lett.} \textbf{110}, 195501.

\bibitem{Saldin:2009jj}
Saldin, D.~K., Shneerson, V.~L., Fung, R. \& Ourmazd, A., 2009 {Structure of
  isolated biomolecules obtained from ultrashort x-ray pulses: exploiting the
  symmetry of random orientations}.
\newblock \emph{J. Phys.: Condens. Matter} \textbf{21}, 134014.

\bibitem{Saldin:2010bx}
Saldin, D.~K., Poon, H.~C., Shneerson, V.~L., Howells, M., Chapman, H.~N.,
  Kirian, R.~A., Schmidt, K.~E. \& Spence, J. C.~H., 2010 {Beyond small-angle
  x-ray scattering: Exploiting angular correlations}.
\newblock \emph{Phys. Rev. B} \textbf{81}, 174105.

\bibitem{Poon:2013ia}
Poon, H.~C., Schwander, P., Uddin, M. \& Saldin, D.~K., 2013 {Fiber Diffraction
  without Fibers}.
\newblock \emph{Phys. Rev. Lett.} \textbf{110}, 265505.

\bibitem{Kurta:2012cb}
Kurta, R.~P., Altarelli, M., Weckert, E. \& Vartanyants, I.~A., 2012 {X-ray
  cross-correlation analysis applied to disordered two-dimensional systems}.
\newblock \emph{arXiv} .

\bibitem{Kurta:2013to}
Kurta, R.~P., Altarelli, M. \& Vartanyants, I.~A., 2013 {X-ray
  cross-correlation analysis of disordered systems: potentials and
  limitations}.
\newblock \emph{arXiv} .

\bibitem{Kirian:2011bq}
Kirian, R.~A., Schmidt, K.~E., Wang, X., Doak, R.~B. \& Spence, J. C.~H., 2011
  {Signal, noise, and resolution in correlated fluctuations from snapshot
  small-angle x-ray scattering}.
\newblock \emph{Phys. Rev. E} \textbf{84}, 011921.

\bibitem{data}
Mendez, D., Lane, T.~J., Ratner, D. \& Doniach, S., 2013, {
Correlated x-ray scattering dataset, silver nanoparticles}.
\newblock \emph{Harvard Dataverse Network [http://dx.doi.org/10.7910/DVN/23244]}

\bibitem{Cohen:2002jw}
Cohen, A.~E., Ellis, P.~J., Miller, M.~D., Deacon, A.~M. \& Phizackerley,
  R.~P., 2002 {cryocrystallography papers}.
\newblock \emph{J. Appl. Cryst (2002). 35, 720-726
  [doi:10.1107/S0021889802016709]} pp. 1--7.

\bibitem{Levard:2011bx}
Levard, C., Reinsch, B.~C., Michel, F.~M., Oumahi, C., Lowry, G.~V. \& Brown,
  G.~E., Jr., 2011 {Sulfidation Processes of PVP-Coated Silver Nanoparticles in
  Aqueous Solution: Impact on Dissolution Rate}.
\newblock \emph{Environ. Sci. Technol.} \textbf{45}, 5260--5266.

\bibitem{Henke:1993wx}
Henke, B.~L., Gullikson, E.~M. \& Davis, J.~C., 1993 {X-Ray Interactions:
  Photoabsorption, Scattering, Transmission, and Reflection at E= 50-30,000 eV,
  Z= 1-92}.
\newblock \emph{Atomic data and nuclear data tables} .

\bibitem{Schroder:2010cm}
Schr{\"o}der, G.~F., Levitt, M. \& Brunger, A.~T., 2010 {Super-resolution
  biomolecular crystallography with low-resolution data}.
\newblock \emph{Nature} \textbf{464}, 1218--1222.

\bibitem{Neutze:2000ih}
Neutze, R., Wouts, R., van~der Spoel, D., Weckert, E. \& Hajdu, J., 2000
  {Potential for biomolecular imaging with femtosecond X-ray pulses.}
\newblock \emph{Nature} \textbf{406}, 752--757.

\bibitem{Spence:2012eo}
Spence, J. C.~H., Weierstall, U. \& Chapman, H.~N., 2012 {X-ray lasers for
  structural and dynamic biology}.
\newblock \emph{Rep. Prog. Phys.} \textbf{75}, 102601.

\end{thebibliography}

{\bf Figure captions}

Figure 1: 
Experimental setup: a kapton capillary filled with a solution of silver NPs (face-centered-cubic). Bragg rings $q_{111}$ and $q_{200}$ are illustrated by circles on the detector plane. At least one of the exposed NPs happens to be oriented such that two reciprocal lattice (body-centered-cubic) peaks are intersecting the detector at $q_{111}$. Dashed lines represent the scattering vectors (separated by the angle $\psi$), and solid lines represent the projection of those vectors onto the detector plane (separated by the angle $\Delta$). Artwork courtesy of Gregory M.~Stewart (SLAC).

Figure 2:
{\bf a)} Representative measurement of one x-ray snapshot, $n_{s}( q_{111},\phi)$, \textit{i.e.}~the photon counts around the Bragg ring at $q_{111}$. The horizontal bars show the binary cutoff along the ring for each module. Here, the $q_{111}$ ring spans 10 modules in total. {\bf b)} Histogram of the intensities in {\bf a} on a log scale of photon numbers. Notice the tail at higher intensities, indicating the presence of large particles. {\bf c)} Result of applying the binary filter to {\bf a}. On average, a binary shot at $q_{111}$ is 10\% ones, 77\% zeros and 13\% masked pixels (masked pixels are ignored in the analysis).  {\bf d)} The average binary intensity from a scan of 500 snapshots. Notice systematic structure in the intensity response, indicating intra-module pixel variations. {\bf e)} Typical snapshot showing the Bragg rings at $q_{111}$ and $q_{200}$. Low/high intensities shown in white/black.

Figure 3:
{\bf a)} From top to bottom, measured correlation functions $D (q_{111},q_{200}, \Delta  )$, $D (q_{111},q_{111}, \Delta  )$, and $D (q_{200},q_{200}, \Delta  )$ from 20 nm silver nanoparticles. We truncated the angular range to highlight the correlation peaks. Regions not shown contain artifacts similar to those on the figure, with nothing greater in magnitude than the CXS peaks. {\bf b)} Corresponding simulations of the correlations plotted in {\bf a}. Vertical lines mark analytical predictions. {\bf c)} Simulation (dashed line) and measurement (diamond marker) of a correlation peak width. The simulation was for 20 nm particles. Peak width scales inversely with particle size, hence we expect the measured CXS resulted from particles larger than 20 nm. Shading represents 95\% confidence intervals ($\pm$ 1.96 $\times $ standard error) on the measurement.

Figure 4:
Shows the shot-to-shot variance in the correlation function (\ref{angular_shot}). Large variance is due to the presence of large particles. We illustrate the effectiveness of the binary filter at removing systematic error induced correlations, even when the large particle fluctuations are minimized. {\bf Left)} Histogram of $D_s(q_{111}, q_{111}, \Delta = \Delta_1)$ over all shots on a log scale (raw data without any filter). Outliers were observed with values up to order 1e8, and were removed (not shown). The large tails indicate the presence of large crystals in the data which can contribute to correlations. Red dashed lines mark $\pm$ 200k counts$^2$. {\bf Right)} Plots of $D(q_{111}, q_{111}, \Delta)$ for filtered and unfiltered data. The average was taken over the 4208 shots where both of the following criterion were met: $-200k $ counts$^2 \le D_s(q_{111}, q_{111}, \Delta = \Delta_1) \le 200k$ counts$^2$ ; $-200k $ counts$^2 \le D_s(q_{111}, q_{111}, \Delta = \Delta_2) \le 200k$ counts$^2$. In the unfiltered plot, artificial correlations due to detector and sample anisotropies are similar in magnitude to the true correlation signal. The vertical blue lines mark the analytical prediction of CXS signal ($\Delta_1$ and $\Delta_2$).

{\bf Short title for page headings}

Correlated X-ray Scattering

\end{document}






